\chapter{Architecture / Implementation}
\label{chapter:architecture}

\todo{schema cu partitionarea si flow-urile}

So far, the HW specifications and features of this router were covered. This
chapter provides details in the internal functioning of the machine, 
how all the services, settings and systems work together in order for 
it to function as a router.

\section{Overview}
\label{sec:general-view}

From a top down perspective the router is composed of 2 different virtual
machines that run on the hardware, that will pe called partitions:
\begin{itemize}
	\item The Master partition
	\item Demilitarised or Slave partition (DMZ)
\end{itemize}

Each parition runs its own kernel and rootfilesystem and is completely
independent of the other partition, meaning it can be shut down or rebooted
without impacting the other partitions. In practice, the DMZ is dependent
on the Master partition, as all traffic that is destined for the DMZ passes
through the Master.

Both Master and DMZ run the same kernel image which is based on the
8.12 linux kernel.The Master partition has its root filesystem loaded
from the HDD, while the DMZ has the root filesystem in flash.

\subsection{Master partition}
This partition is responsible for handling most of the HW the machine has.
It controls 3 of the 4 cores, the HDD, all the conectivity ports and 
network interfaces.

It also functions as a gateway, linking the external network (the internet)
with the Wireless, LAN and DMZ.

As such, it's natural for all the critical software components and services
to be hosted on the Master partition.

\subsection{DMZ partition}
The second partition acts as a deployment space for all the services that need
to be accesed from the external network. This partition can be compromised
without endangering the router or the internal network.

Because the root filesistem is always loaded from flash memory, the DMZ
acts as a frozen machine, any changes made there being non-persistent.
In case of a malitious attack, a simple reboot will restore the machine to
it's previous state, without endangering the Master partition.

The DMZ hosts a lighttpd server which serves static content, but virtually
any service that can run on a single core can be deployed there.
In case of more complex servers (which need persistent storage) the DMZ
cand act as a frontend, having a backend on the Master partition and exchanging
calls.

\section{Networks}
\label{sec:networks}

\todo{schema cu toate retelele}
From the networking point of view, the router hosts 4 networks and 3 addres spaces:
\begin{itemize}
	\item LAN + Wireless are bridged under the same address space
	\item External network provides the Master partition with a dinamic IP, making
	it a gateway
	\item Communication between partition is done through macless ports
\end{itemize}

\subsection{The Bridge}
\todo{schema cu bridge-ul si switch-ul}
The wireless interface(wlan0) and the LAN interface(eth1) are bridged using
OpenVSwitch, an open source solution for software switches. The bridge
acts as contact point between LAN, Wireless and the other networks.
The bridge interface is configured with a static IP and the DHCP server
listens for request on this interface.

Switching inside the LAN network is done at high speed, as the hardware switch
handles all the packets sent between the physical switch interfaces, but all
traffic that is sent to the Wireless network has to go through the bridge and
through one of the CPUs.

The switch has an internal port connected to the machine (controlled by the
operating sistem) fig??? which is used as an exit point for all the internal
traffic of the switch. The problem is that this port is Gigabit,
so it's speed is insufficient in order to service the combined traffic
speeds of the external ports simultaneously. This is effectively a bottleneck.
Such behavior can occur for example when all the machines connected to the 
external ports try to transfer data to the internet.

The Wireless Adapter has a maximum speed of 150Mbps so the maximum 
(theoretical) transfer rate between a machine connected through 
the Wireless network is limited to that amount.
The setup can be upgraded with a  more powerful wireless card, 
but for this type of application the Gigabit Wireless card was considered
unnecessary.

\subsection{Connecting to the Internet}
One of the Ethernet ports(eth0) is connected directly to the exterior network.
By default this internet receives it's IP dinamically, but it can be
configured to have a static IP, or to connect through PPPoE.

Being the only connection with the exterior, this makes the Master Partition
the Gateway of all the networks. With this in mind, certain protocols and
services such as NAT and firewalling are specific only to this interface.

In a real-life scenario, where several LAN clients are trying to send
traffic to the exterior this interface can prove to be a bottleneck, as 
it's speed is Gigabit(1Gbps), while the combined maximum traffic of all the
other networks exceeds the ports capabilities.

\subsection{Communicating with the DMZ}
The communication between the DMZ and the Master partition is done with the
help of the hypervisor with a pair of preconfigured macless ports.
From the operating sistem point of view, these macless interfaces behave
identically with other interfaces, the only difference is in their
implementation in the kernel.

The macless ports are configured statically, the Master having the address
1.1.1.1 and the DMZ having 1.1.1.2. Int he current setup, this connection
is used mainly for controlling the DMZ, as the Slave partition has no means
of connecting to the Master in this way. The motivation behind this design
choice was the total isolation of the DMZ in case of an attack.

\section{Services}
\label{sec:services}

\todo{Spune despre toate serverele instalate si ce fac}

The machine needs certain servers and services in order to function as a 
router. This section provides details on the installed services that are
currently available how they are configured.

\subsection{Master Partition services}
The Master partition needs to bridge the LAN with the Wireless network.
This is done using OpenVSwitch. Both interfaces are aggregated unde a
virtual interface, br0.

Listening on \textbf{br0} are certain servers:
\begin{itemize}
	\item Hostapd server, for authentification on the Wireless network.
	\item DHCP server, servicing any IP requests. In order fo the DNS to work
	relaying is configured in the DHCP server, informing any new connected
	machines of the address of the external DNS server. This is currently
	done statically, therefore changing the ISP, or connecting the router
	to a new external network implies changing the mentioned DNS address
	in the dhcpd configuration file.
	\item SSH server, for remote connection and configuration
	\item Webmin server, having increased importance, allowing
	remote connection and configuration using any web browser in order
	to provide conveninence in administration.
\end{itemize}

No specific firewall rules are applied to the \textbf{br0} interface, except
for a forwading rule, that diverts all incoming traffic on port \textbf{8080} 
to port \textbf{80} on the \textbf{DMZ}. This is the only way to acces 
the webserver that is running on the Slave Partition.

The Gateway interface, \textbf{eth0} has a limited number of servers servicing
it, but it has the most firewall rules.

The only servers listening on this interface are the \textbf{SSH} server, for
configuration and the \textbf{DHCP} client for obtaining the dynamic IP.
On demand, a PPoE client can be configured,.

All external traffic for ports greater than \textbf{2000} is blocked by the HW firewall,
so packets destined for other ports are dropped without reaching the SW.
The exception is port \textbf{8080}, which is configured to be forwarded 
to port \textbf{80} on the DMZ in a similar way to the \textbf{br0} interface.

As for \textbf{NAT}, necessary for communicating between the internal and
the external network, it is available through \textbf{iptables} rules.

\subsection{DMZ services}
The DMZ services are limited compared to the Master Partition.
There is a \textbf{SSH} server running, for remote administration and
a \textbf{lighttpd} server that serves static content.

Howerver, almost any server that can be effectively run on a single core
machine can be hosted here, for example FTP, streaming and mail servers.

Due to the nature of the root filesistem setup on the DMZ, installing new
servers implies making modifications to the rootfs and then writing the
new rootfs to the flash memory. Servers also need to be able to recover
gracefully in case the DMZ encounters a hard reset.


\section{Packet flows}
\label{sec:flows}

Through design, the router has a number of different packet flows.
Understanding and separating them can help in a number of applications,
for example setting different processor affinities based on the 
type of incoming traffic.

In figure?? 4 packet flows are identified. Although one can argue that more
flows can be isolated, they are, in general, separated by their different
source and destination.

The first flow, marked in COLOR1 in diagram??? is the LAN/Wireless flow.
Traffic is generated in either Wireless or LAN and it has it's destination
in the same network. This includes LAN to LAN, Wireless to Wireless and 
LAN to Wireless traffic. This traffic is not subject to any artificial
limitations or filtering, and the speed is limited by HW.

The typical usecase that generates this kind of traffic is data transfer
between computers on the internal network (in the same address space).

The second flow, marked in COLOR2 is between the internal network and the
external one. Traffic on this flow is filtered, external ports greater
than \textbf{2000} being closed, with exceptions. Because the 2 networks
are in distinct address spaces, all packets need to pass through NAT.

A tipical usecase that would generate this flow is a computer on the internal
network trying to access the internet.

The third and fourth flow are similar, each having one endpoint in the DMZ
partition. They are generated when either the external network or the internal
one tries to acces resources located on the DMZ. These resources are not
accessed directly, as the interface between the two partitions is not
directly addressable by the two networks.

For example when a machine on the internal network wants to acces the http
server located on the DMZ, it will access port \textbf{8080}
on its gateway address (in this case, the bridge, \textbf{br0}, interface).
Incoming connections on port \textbf{8080} are routed statically to port
\textbf{80} on the DMZ machine.

The same principle applies for accesing the DMZ located http server from
the external network. A request to port \textbf{8080} of the external
interface are routed to the DMZ on port \textbf{80}.

Any server running on the DMZ that needs to be access receives it's own
forwarding rule, statically.
