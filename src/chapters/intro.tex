\chapter{Introduction}
\label{chapter:intro}

Routers are dedicated netwoking devices, specialized in handling data
packets between computer networks. Their main purpose is forwarding packets to
their next destinations. When a data packet reaches a router through one of
the lines, the destination address is read and then, based upon a routing
table or specialised routing policies and algorithms, it is sent toward
its next destination.

Since the apparition of routers in the mid '70s, their role has not 
fundamentally changed, but as the hardware and the software have advanced, 
routers have become more compact, affordable and the offered functionalities
have diversified.


\section{State of the Art}
\label{sec:state-of-the-art}

Early routers were general purpose computers configured to forward packets
between networks. With time, they were replaces by specialised hardware,
that was more poweful in handling network packets.
Nowadays, routers incorporate internal memory, general purpose processors, 
switches and wireless capabilities.

Advanced routers host tens of network interfaces, Harddisks, multiple cores,
proprietary operating systems and range of communication processors
for offloading different tasks in hardware (Security, pattern matching etc)

\section{Context and Motivation}
\label{sec:context-and-motivation}

High end routing equipments have a broad range of configuration options
available but the price can reach several hundred dolars. At the other
end of the spectrum, low end routers offer a reduced number of network
interfaces and poor performance.

For casual users, a low end router can fulfill all their needs, but as
demands go up, so does the price. Advanced users and small companies
are interested in more configuration options but are reluctant to invest
in high end routers. 

This "Router on a chip" addresses the needs of medium/advanced
users and small companies who need more options, decent performance and
maybe hosting space without making a pricey aquisition.

In chosing hardware for this router several factors were followed:
\begin{itemize}
	\item The processor had to be multicore and had to be able
	to support an Unix based operating sistem
	\item The router had to have similar networking performance, with
	it's comercial counterparts
	\item Power consumption had to be as low as possible
	\item Price had to be as low, compared to high end devices
\end{itemize}

As it matched most of the requirements, the T1040 processor provided by
Freescale was chosen for building this router.
