\chapter{Introduction}
\label{chapter:intro}

\textbf{This is just a demo file. It should not be used as a sample for a thesis.}\\
\todo{Remove this line (this is a TODO)}

\section{Project Description}
\label{sec:proj}

\section{Background}
\label{sec:background}

This thesis presents the \textbf{\project}.


\section{The Problem}
\label{sec:the-problem}


Lorem ipsum dolor sit amet, consectetur adipiscing elit. Aenean aliquam lectus vel orci malesuada accumsan. Sed lacinia egestas tortor, eget tristiqu dolor congue sit amet. Curabitur ut nisl a nisi consequat mollis sit amet quis nisl. Vestibulum hendrerit velit at odio sodales pretium. Nam quis tortor sed ante varius sodales. Etiam lacus arcu, placerat sed laoreet a, facilisis sed nunc. Nam gravida fringilla ligula, eu congue lorem feugiat eu.

\section{The Solution}

Lorem ipsum dolor sit amet, consectetur adipiscing elit. Aenean aliquam lectus vel orci malesuada accumsan. Sed lacinia egestas tortor, eget tristiqu dolor congue sit amet. Curabitur ut nisl a nisi consequat mollis sit amet quis nisl. Vestibulum hendrerit velit at odio sodales pretium. Nam quis tortor sed ante varius sodales. Etiam lacus arcu, placerat sed laoreet a, facilisis sed nunc. Nam gravida fringilla ligula, eu congue lorem feugiat eu.



Introduction:
-length= ??(probabil pana in 4 pagini)    

* Background:
        - state of the art
        - utilitatea routerelor de genul
        - context / routere / ce exista pe piata
        - cam 1 pagina?
        
    * The problem
        - ce probleme au solutiile existente
        - 0.5 - 1 pagini?
        
    * The solution
        - ce fac eu
        - ce am in plus, de ce e mai bun
        - foarte sumar
        - cam 1 pagina


Main body:
    -length= ? (20-30 pagini)

* The Freescale Platform
        - hw specs
        - detalii generale arhitectura
- ce face bine, de ce e bun pentru un astfel de proiect
- 4 pagini?        

    * Characteristics
        - ce functionalitati are
        - prezentat in subcapitole separate fiecare parte
        (TODO)
        - poate intra mult, deci poate chiar 10

    * Architecture:
        - scheme
        - altceva???


    
* Scenarii de folosire:
        - small office router    
        - lightweight hosting
        - TODO: mai gaseste chestii de bagat
    
    * Performante:
        - maybe?
