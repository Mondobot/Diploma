\chapter{The T1040 platform}
\label{chapter:t1040-platform}

In order to create a proper router setup one needs to understand
all the hardware specifications of available products and select the
most suitable hardware.

The decision to use the T1040 platform can be explained by examining the
hardware specifications and its general architecture, which is the aim of
this chapter.

Previously, certain requirements for the finished router
were stated. Proving to be a suitable platform for this router
implies catering to all these needs. Simply listing the available
hardware is not enough as this chapter needs to explain \textbf{why}
the hardware caters best to the needs of the router.

The first part presents general hardware specifications,
necessary when comparing this platform to other commercial router
hardware. In the other sections, advanced construction and architectural
features are reviewed, as these provide imporant advantages for the 
functioning router.

\section{Hardware specifications}
\label{sec:specs}

For a complete list of hardware specifications see [\footnote{T1040 HW specs -
http://www.freescale.com/webapp/sps/site/prod_summary.jsp?code=T1040}].

\subsection{CPU}
One of the necessary requirements for the processor to have multicore 
architecture and to be strong enough to support a full fledged Unix
based operating system.

In terms of CPU processing power, the platform hosts four processing cores
running at a frequency of up to 1.4Ghz each. These small yet powerful e5500
\footnote{http://www.freescale.com/webapp/sps/site/overview.jsp?code=64BIT}
are based on the Power architecture are equipped with 256KB L2.
The processing power is more than enough in order to run a different operating
system on each core with all kinds of servers and services.
While it doesn not have graphical memory, the setup can even run
graphical servers, although with reduced performance.

Being powerful enough to run multiple operating systems simlutaneously,
not using virtualization would greatly limit the potential of this
platform. Running several virtual machines is usefull for this setup,
but performance can be hindered if virtualization is handled by software.
When virtualization is done in software, virtual machines are treated
like normal programs by the operating system. Unfortunately, 
accessing hardware devices from the virtual machine(VM) implies
passing the call from the VM to the operating system, which in turn is able
to access the hardware. This creates overhead and the VM performance is
rather limited.

Hardware virtualization solutions have increased complexity,
but they offer improved performance over software solutions.

An important feature of this platform is the presence of the 
3 levels of instructions: user, supervisor and hypervisor.
Instruction levels in CPU instruction sets control access of the program
currently running on the processor to resources such as I/O ports, special
instructions and memory regions. The hypervisor instruction level
is used in hardware virtualization by the Virtual Machine Manager(VMM)
or hypervisor, that acts as a host for the virtual machines.

All these features allow the platform to run multiple VMs without
losing performance, greatly extending application scenarios that can be run.

\subsection{Memory}
Random Access Memory(RAM) is important in every machine, as its speed and
quantity shape the overall performance of the machine. Too little RAM
and programs running at one time on the machine are limited both in
performance and number. Having an outdated type of RAM implies slow
access speeds which again, limit the overall performance of running tasks.

In the virtualization environment RAM requierements are even greater,
as each machine needs its own dedicated RAM space.

The T1040 platform supports DDR3 type RAM and the maximum throughput is
1600MT/s. For this application only 2GB of RAM were installed, as this is
satisfactory for the type of applications running.

As this is a networking application Direct Memory Access(DMA) is one
of the critical features used by the system. DMA allows subsystems 
to access the main system memory independently of the CPU.
To address this, the platform hosts dual four channel DMA.

\todo{reread the above paraghraph}

On the connectivity side, included are 2 Serial ATA(SATA 2.0) controllers,
enhanced secure digital host controllers (SD/MMC/eMMC), 2 USB2.0 ports
with integrated PHYs, 4 PCI-express ports, controllers for NAND and NOR
flash memory and 4 UART ports.

\todo{Poate e mai bine ca lista??}

The networking connectivity includes 5 Gbps Ethernet MAC ports (with support for
SGMII and QSGMII interfaces) and a hardware Gigabit Ethernet switch with 8 ports.

\todo{Spune de specificatiile placii wireless}
\section{General arch details}
\label{sub-sec:arch-details}

Being a Communications Processor T1040 offers facilities for speeding up
packet analysis, clasification and distribution, by offloading 
them in hardware. Also present are Buffer Managers and Queue Managers 
for handling frames in hardware, before they get sent for processing.

%\labelindexref{Figure}{img:t1040-diagram}.
\todo{cred ca e mai bine pe lateral, pusa pe o pagina intreaga}
\fig[scale=0.5]{src/img/t1040-diagram.jpg}{img:t1040-diagram}{T1040 diagram}

As soon as a packet enters through the physical interface it gets handled 
by a Frame Manager. The Frame Manager is responisble for having pre-allocated
frames for the incoming packet. This is where the Parse, Classify, Distribute
step takes place. The pattern maching is done by a configurable HW engine.
Based on the analysis, the packet is then classified and handed to the
Queue Manager. The Queue Manager can set affinities for different types of
traffic, allowing to send all packets of a type to preset processor

We can also have citations like \cite{iso-odf}.

\section{Why is it good for what we need}

The presence of a Frame Manager, PCD(Parse, Classify, Distribute) engine
and encription/decription engine means that an important part of 
the operations done by a router, for example static forwarding, 
blocking ports and filtering traffic can be done by the hardware.

All this hardware offloading capabilities of this platform helps save 
important CPU cycles allowing it to maintain higher transfer rates with with
less CPU load, leaving more capabilities for the applications running on the
machine. 

These advantages recommend the platform for this type of application.
