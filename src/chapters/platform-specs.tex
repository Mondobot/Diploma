\chapter{The t1040 platform}
\label{chapter:t1040-platform}

This chapter offers details on the hardware used in this "Router on a chip". The
first section offers hardware specifications necessary when comparing this
platform to other commercial routers, the second section offers general details
on the specific architecture used when building the platform, and the last
section evaluates the strengths and weakneses of the hardware in order to asses
better the final performance of this application


\section{HW specs}
\label{sec:specs}

The t1040 platform hosts a quad core processor and targets the low-end sector through
it's price accesibility and low power consumption. 
The e5500 cores are based on the Power architecture, have a maximum frequency of
1.4Ghz and host a 256KB L2 cache each. 

An important feature for this aplication
is the presence of the 3 levels of instructions: user, supervisor and
hypervisor. This allows the processor to cooperate with a hypervisor, enabling
hardware virtualisation and extending the application scenarios that can be run.

The RAM memory is DDR3 and the platform support a maximum throughput of
1600MT/s. DMA is dual four channel

\todo{reread the above paraghraph}

On the connectivity side, included are 2 Serial ATA(SATA 2.0) controllers,
enhanced secure digital host controllers (SD/MMC/eMMC), 2 USB2.0 ports with integrated PHYs, 4
PCI-express ports, controllers for NAND and NOR flash memory and 4 UART ports.

\todo{Poate e mai bine ca lista??}

The networking connectivity includes 5 Gbps Ethernet MAC ports (with support for
SGMII and QSGMII interfaces) and a hardware Gigabit Ethernet switch with 8 ports.


\section{General arch details}
\label{sub-sec:arch-details}

Being a Communications Processor T1040 offers facilities for speeding up packet
analysis, clasification and distribution, by offloading them in hardware

%\labelindexref{Figure}{img:t1040-diagram}.
\todo{cred ca e mai bine pe lateral, pusa pe o pagina intreaga}
\fig[scale=0.5]{src/img/t1040-diagram.jpg}{img:t1040-diagram}{T1040 diagram}


We can also have citations like \cite{iso-odf}.

\section{Why is it good for what we need}


