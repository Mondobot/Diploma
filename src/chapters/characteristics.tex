\chapter{Router characteristics}
\label{chapter:characteristics}

This chapter presents the different functionalities of the router.
The features range from basic requirements, needed in order
for the setup to function as a minimal router to advanced features,
that separate this setup from other comercial routers.
All the features presented are for \textbf{this} setup.
There are other functionalities that can be added, as the hardware
and software allows it, but these are just a proof of concept.

\section{Basic functionality}
These functionalities are needed by the application in order to function
as a router. They are provided by almost all commercial routers,
therefore they can not be absent.

\subsection{LAN network with switching}
\label{sub-sec:lan-switching}
One of the basic characteristics of a router is being able to function as
a switch. Switching involves passing packets between different LAN segments.
This can be done either at Layer 2 of the OSI reference model, using MAC
addresses or at Layer 3, using IP addresses. Switches employ a number of
protocols in order to determine where to send each incoming packet.

Switching can be done both in hardware and in software.
Software solutions, while being cheaper and more versatile
in configuring, have a lot to suffer when it comes to intense 
packet crunching. This is why HW switches are preffered.

This setup integrates a hardware Layer 2 switch with 8 LAN ports 
and is able to handle the packets at speeds up to 1Gbps.

\subsection{Wireless network}
\label{sub-sec:wireless}
In recent years, Wireless capabilities have become one of the most
used features of routers. With the increase in popularity of mobile devices, laptops,
tablets and smartphones which rely almost solely on this technology in order to connect
to the internet, not having a wireless network integrated into the router
is not an option. Furthemore the adapter needs to be compatible with 
all kinds of devices, both new and old.

To address this need, this router has a wireless network card that complies
with the IEEE 802.11b/g/n standards and has a transfer rate of up to
150Mbps.

\subsection{Wireless security and authentification}
\label{sub-sec:wireless-sec}
Simply having a wireless connection is not enough for the good
functioning of the router. Security is a very important issue.
Being able to limit access to the wireless network only to the
authorised devices is crucial.

Therefore, the setup has to offer authentification support. Out of
several alternatives for an authenthification service,
HostAP was chosen as it is easy to use and does not need constant
configuration.

Out of the authentification protocols, WEP,WPA and WPA2, the latter was chosen
for its improved security over the former two.

\subsection{Wireless and LAN bridge}
\label{sub-sec:bridging}
On basic routers that feature both LAN and Wireless support, devices
connected through different networks need to be able to communicate with
each other.

With this in mind, LAN and Wireless are bridged forming a single address space.
This choice was made for the sake of simplicity, although it has a downside.
Switching between LAN and the Wireless network is limited in speed compared 
to switching between the LAN ports.

\subsection{HW/SW routing}
\label{sub-sec:routing}
Routing, along with switching represents the core of the router's functions. 
This device currently has capabilities for routing both in software
and using the hardware.

Routing in software relies on setting static routes in the operating system.
It is easily configurable, but the disadvantage is that all these operations
consume precious CPU cycles. Therefore, routing using the hardware is
preffered, where possible.

Hardware routing implies configuring the Frame Manager to intercept
certain types of traffic / addresses, to bypass verification by the kernel 
and send them directly to a specified port. While it is a bit rigid and
hard to configure, it can provide a very important speed boost over software
routing.

\subsection{Network Address Translation}
\label{sub-sec:nat}
Because the IPv4 address space is limited, routers usually have just one
address, as it seen from the Internet. If each machine serviced
by the router would have a different external IP address, the whole
IP space would be used up very fast. The problem is that most users
want to connect multiple machines to a router and each of them has to have
internet access. They each have their own internal IP address but they
must share only one external IP address.

In order to solve this problem, the Network Address Translation protocol (NAT)
was created. NAT holds mappings between internal IP addresses and external 
ports. Each device that wants to access the internet will receive its own
port on the external interface which will service all traffic directed
to that machine. From the internet's point of view, there is only one
machine, but multiple ports.

This requirement had to be addressed as well, as the absence of NAT would
mean that the internet is accessible only from the router, not from
the connected devices. Therefore, NAT is done in software, through 
\textbf{iptables}.

\subsection{SW services}
\label{sub-sec:sw-services}
Modern roters employ a number of software services in order to run
efficiently. Amongst these, there are DHCP client and server,
PPPoE client and port forwarding.

DHCP (Dynamic Host Configuration Protocol) is a standardized networking
protocol used for distributing network configuration parameters such as
IP addressesi automatically. When using DHCP, devices request networking parameters
from a DHCP server, removing the need for a network administrator to
configure these settings by hand.

The router needs both a DHCP client and a server. The client is used in
order to request networking paramenters from the Internet Service Provider 
(ISP) DHCP server. The internal DHCP server is used to service requests coming
from the internal network, LAN and Wireless.

The Point-to-Point Protocol over Ethernet (PPPoE) is a network protocol
for encapsulating PPP frames inside Ethernet frames. It currently is one
of the most used methods of connecting a device the ISP network.
It's an alternative way of connecting and needs to be offered as an option,
in order to cover basic usecases of the router.

The last basic service offered is Port Forwading. Port forwarding
binds an external port,located on the router,to a port located on a machine
in the internal network. All traffic sent to the specified
external port is forwarded to the internal port. As some services
are port dependent, NAT can change incoming ports so these services
can't function properly. This technique is used to permit communication between 
external hosts and services provided within the private LAN.

All these protocols and servers are available on the router. DHCP client
and server are installed, the PPPoE client is installed and port forwarding
is done through \textbf{iptables}.

\subsection{Easy administration using Webmin}
\label{sub-sec:webmin}
Configuring a router can be a confusing task, especially for inexperienced
users. Although this router targets medium/advanced users, the problem
of easy configuration still remains. Normally, this setup would be configured
by connecting through SSH and by changing different options and configuration
files directly. However, a more elegant and accesible solution can be found.

Routers usually can be configured through a web interface. There are some
clear advantages with this approach. The first one is that one can do the configuration
on any device that has a web browser and is connected to the network.
The second  advantage is that all options are a few clicks away, easy to find and
alter.

With this in mind, a router that is not capable of offering easy configuration
would have a big dissadvantage over comercial routers. As a solution,
a Webmin server has been installed on the router.

Webmin\footnote{http://www.webmin.com/} is an open source project that provides a web-based interface for
system administration for Unix. Using any modern browser one can setup
user accounts, Apache, DNS and change everything you want about the
administrated machine.

Using Webmin administrators don't need to connect physically to the machine
and manually edit configuration files. It also provides support for scripts,
a shell and has a modular construction, making it easy to extend or limit
the functionalities of the client.

Webmin is even more powerful than normal web-based router configuration
solutions, as it allows its users to change virtually any setting on the
router machine.

\section{Advanced functionality}
Basic features are the core of the router, but the advanced feature
separate this router setup from other commercial solutions and provide
powerful aditions to the router capabilities.

\subsection{HW/SW firewall}
\label{sub-sec:hw-firewall}
The firewall is a security system that controlls incoming and outgoing
network traffic based on specific rules. The firewall is a barrier 
between the trusted, internal network and the external network, deemed
unsafe.

Software firewalls are very customisable but costly for the CPU and can
lead to performance deterioration. On the other hand, hardware firewalls
don't put any stress on the CPU, but are harder to configure.

The router has firewalling capabilities both in software and hardware.
Software firewalling is done using \textbf{iptables}. Hardware firewalling
is enabled by configuring the Frame Manager. This HW firewall can drop
traffic, block ports and filter traffic based on source and destination.

\subsection{Demilitarized partition}
\label{sub-sec:dmz}
The feature that makes this setup stand out the most compared to similar
performing routers is the presence of the virtualisation environment,
and virtual machines.

The Demilitarized Partition(DMZ) is a virtual machine, running on the 
router that can be used to deploy all sorts of applications, for example
web hosting, email servers and FTP servers.
It is a full fledged machine, having its own dedicated CPU core, 
kernel image and filesystem and most importantly it can be restarted
without restarting the whole router.

This DMZ partition can be compromised without losing data, or contaminating
the entire machine. The root filesystem is loaded from flash, 
but it can be loaded restored from an virtual machine image as well.

Having such a feature is a powerful bonus of this router as the DMZ cand
host almost any application that runs properly on a single core machine.
It is secure, completely isolated, easy to setup and restore.

Installed on the DMZ partition is a lighttpd server that serves static
content. It is just an example of the applications that can run securely,
without impacting the router's overall performance.

\subsection{Storing space on the HDD}
\label{sub-sec:storing}
Routers usually feature limited storing space, but in recent years, this
feature has become more and more requested. Having a full HDD usually
increases the router price beyond mid level but for multiple potential
virtual machines and for the type of applications that the router
is able to run, not having enough storing space can be detrimental

As a result the router is equipped with a 250GB HDD. All applications
benefit from this storing space and it can even save multiple snapshots of
the DMZ parition.
