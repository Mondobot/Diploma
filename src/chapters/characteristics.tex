\chapter{Features}
\label{chapter:features}
\todo{Capitolul este complet, dar continutul trebuie extins}

This chapter presents the different functionalities of the router.
Some of them are basic, while some of them are more advanced.
The basic features are needed in order for it to function as a router,
while the advanced ones showcase some interesting applications that
can be housed and run on this router.

\section{Basic functionality}
These functionalities are needed by the application in order to function
as a router. They are provided by almost all the commercial routers,
therefore they can not be absent.

\subsection{LAN network with switching}
\label{sub-sec:lan-switching}
One of the basic characteristics of a router is being able to function as
a switch. Software solutions, while being cheaper and more versatile
in configuring, have a lot to suffer when it comes to intense 
packet crunching. This is why HW switches are preffered.

The integrated hardware switch provides 8 LAN ports and the chip is able
to handle the packets up the Line Rate of 1Gbps.


\subsection{Wireless network}
\label{sub-sec:wireless}
Included in the router is a wireless network that complies with the
IEEE 802.11n standard.

\todo{mai spune ceva?}

\subsection{Wireless security and authentification}
\label{sub-sec:wireless-sec}
Limiting access to the wireless network is done in software by hostapd,
through the WPA2 protocol.

\subsection{Wireless + LAN bridge}
\label{sub-sec:bridging}
LAN and Wireless are bridged forming a single address space. This choice
was made for the sake of simplicity, although it has a downside. Switching
between LAN and the Wireless network is limited in speed compared 
to switching just between the LAN ports

\subsection{HW/SW routing}
\label{sub-sec:routing}
Routing is the core of the application. It can be done both in HW and SW.

Routing using the hardware means configuring the Frame Manager to intercept
certain types of traffic / addresses, to bypass verification by the kernel 
and send them directly to a specified port. While it is a bit rigid and
harder to configure, it can provide a very important speed boost over SW
routing.


\subsection{NAT}
\label{sub-sec:nat}
\todo{Oare trebuie explicat ce este NAT??}
Network address translation is done in software, through iptables.

\subsection{SW services}
\label{sub-sec:sw-services}
Among the basic software services there is a DHCP server, responsible 
for servicing the LAN + Wireless.

Port forwarding thorugh iptables.

\todo{oare NAT merge pus tot aici?}

\subsection{Easy administration using Webmin}
\label{sub-sec:webmin}
\todo{da linkuri si explica mai bine ce e Webmin}
Webmin is an open source project that provides a web-based interface for
system administration for Unix. Using any modern browser one can setup
user accounts, Apache, DNS and change everything you want about the
administrated machine. 

Using Webmin administrators don't need to connect phisically to the machine
and manually edit configuration files. It also provides support for scripts,
a shell and has a modular construction, making it easy to extend, limit
the functionalities of the client.

All administration is done through Webmin.

\section{Advanced functionality}
These features are not critical in order for the machine to run as a router,
but provide powerful aditions to the range of applications the machine has.


\subsection{HW/SW firewall}
\label{sub-sec:hw-firewall}
The router has firewalling capabilities in software, through iptables.
It also has the ability to configure the HW in order to limit specific IPs,
block ports and filter incoming traffic.

\subsection{Demilitarized partition}
\label{sub-sec:dmz}
For improved security, the router features two different virtual machine
partitions: the Master partition, which controlls 3 of the 4 CPU cores
and all the HW, and a Demilitarized partition (DMZ), which controlls just one
CPU core.

This DMZ partition can be compromised without losing data, or contaminating
the entire machine. It can be easily rebooted and restored without 
rebooting the whole router. The root filesystem is loaded from flash, 
but it can be loaded restored from an virtual machine image as well.

It has a range of applications, for example hosting webservers and it's
completely isolated.

\subsection{SW services}
\label{sub-sec:sw-services}
Installed on the DMZ partition is a lighttpd server that serves static content.
It's just an example of the application that can run securely, without
impacting the router's overall performance
