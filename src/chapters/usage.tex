\chapter{Usecases}
\label{chapter:usecases}
In previous chapters both software and hardware capabilities of this
setup were preseted.
There were also indications on potential bottlenecks and performance
limitations. All these characteristics, combined with the price range
of the platform and its power consumption, recommend it for certain
applications. The limited number of interfaces makes this router
better suited for home / small office use. 

The wireless card supports
more connections than the physical ones, but perfomance is dictated
by the type of the wireless card. The currently instaled one is an entry
level card, so performance is rather weak.

Another aspect taken into consideration is that, while easy to configure
through the web interface, the setup offers a great range of configuration
options that can be hard to understand for the average user. With this
in mind, this router is most likely destined for medium and advanced users.

\section{Simple Router}
\label{sec:use-router}
The first usecase for this setup is a simple router, suited for home
or small office use. Basic router features like NAT, DHCP, switching,
wireless as well as the configurable firewall are the base of this usecase.

In most cases, performance in this case is not critical,
the DMZ is not used and the HDD is not necessary.

While it can function perfectly in this role, other single-core solutions
can be found on the market that offer similar perfomance, but feature a
lower power consumption, lower price and are easier to configure and
maintain.

If performance is critical, the DMZ can be disabled (all 4 cores would be
controlled by the Master Partition), the wireless card can be upgraded
and certain CPU affinities can be set in order to ballance the load.

The conclusion of this usecase is that if performance is not critical,
this is a sub-optimal choice for role of simple router.

\section{Lightweight hosting service}
\label{sec:hosting-service}
Another potential usecase of this setup is that of a hosting platform for
several webservers. The features that recommends it for such a purpose
is the customizable number of virtual machines. If router services are
not critical, the master partiton can downgraded and up to 3 virtual
machines can function at the same time on the machine, each with its own
servers. These machines function independently and can be powerd on/off and
reset without disturbing the whole ecosystem.

The usecase is called "lightweight", as this architecture
is not the best suited for heavy server-side processing.

In conclusion, this router can offer some powerful customization options
for small companies trying to host some of their servers on isolated
environments.

\section{Hybrid router-hosting service}
\label{sec:hybrid}
The usecase that is most suitable for this setup is that of a hybrid,
deploying full router capabilities while still having one or two virtual
machines used of services.

This would be the best use of the router's resources and would prove valuable,
cost wise and performance wise. The designation would still be "lightweight"
as the hardware isn't best suited for handling CPU intensive tasks.
